% !TeX spellcheck = ca
\documentclass{article}
\usepackage[utf8]{inputenc}
\usepackage{graphicx}
\usepackage{hyperref}
\usepackage{amsmath}
\usepackage{ amssymb }
\usepackage{tikz}
\usepackage{float}
\usepackage[simplified]{pgf-umlcd}
\usepackage{subfiles}
\usepackage{listings}
\usepackage[lighttt]{lmodern}
\usepackage{color}
\usepackage{array}
\usepackage{textcomp}
\graphicspath{ {img/} }

\usetikzlibrary{positioning,fit,calc,arrows.meta, shapes, snakes}
%Tot això hauria d'anar en un pkg, però no sé com és fa
\newcommand*{\assignatura}[1]{\gdef\1assignatura{#1}}
\newcommand*{\grup}[1]{\gdef\3grup{#1}}
\newcommand*{\professorat}[1]{\gdef\4professorat{#1}}
\renewcommand{\title}[1]{\gdef\5title{#1}}
\renewcommand{\author}[1]{\gdef\6author{#1}}
\renewcommand{\date}[1]{\gdef\7date{#1}}
\renewcommand{\baselinestretch}{1.5}
\renewcommand{\maketitle}{ %fa el maketitle de nou
	\begin{titlepage}
		\raggedright{UNIVERSITAT DE LLEIDA \\
			Escola Politècnica Superior \\
			Grau en Enginyeria Informàtica\\
			\1assignatura\\}
		\vspace{5cm}
		\centering\huge{\5title \\}
		\vspace{3cm}
		\large{\6author} \\
		\normalsize{\3grup}
		\vfill
		Professorat : \4professorat \\
		Data : \7date
\end{titlepage}}
%Emplenar a partir d'aquí per a fer el títol : no se com es fa el package
%S'han de renombrar totes, inclús date, si un camp es deixa en blanc no apareix

\title{SpreadSheet}
\author{Ian Palacín Aliana, Quim Picó Mora, Sergi Simón Balcells}
\date{17 de Maig}
\assignatura{Ampliació de Bases de dades i Enginyeria del Programari}
\professorat{JM. Gimeno Illa}
\grup{Prelab 2}

\renewcommand{\refname}{Referències}
%Comença el document
\begin{document}
	\maketitle
	\newpage
	\section{Introducció}%ian
	\section{Disseny}
	\subsection{Diagrama UML}%InteliJ
	\subsection{LiftA2}%Sergi
	% Operació de haskell
	% La idea es treure la necessitat dels ifs en la Operation (treure la lògica del MaybeValue)
	% S'ha implementat com un double dispatch
	\subsection{Operation i el patró Strategy}%sergi
	% Template method quedava una mica estrany utilitzant liftA2
	% Strategy quedava més net
	\subsection{EitherValue}%Sergi
	% Utilitzar MaybeValue no encapsulava els errors que es poguessin generar de forma
	% dinàmica en Strategy.
	% Un malentès en el disseny de Sheet va fer fer un refactor per a EitherValue.
	% El refactor va ser senzil, ja que liftA2 es aplicable en Eithers (és un aplicatiu)
	\subsection{LazyValue a Cell}%Quim
	% Per a realitzar lazyValue en les cel·les se va fer servir el patró observable. 
	% Cada vegada que es canvia d'expressió es canvien tots els observables per cridar
	% el metode només quan es necessari
	\subsection{Sheet}%Ian
	% La creació de les cel·les es lazy, es a dir, només es creen quan son estrictament necessaries,
	% per a evitar ocupar massa memória en sheets molt grans quan no s'utilitza la majoria de cel·les.
	% Tot i que accedir amb una Array d'arrays sembla més natural, un hashmap es molt transparent
	% per com l'utilitza l'usuari final
	\section{Problemes}
	\subsection{References}%Ian
	% References calculaba referències de la seva referència, i això donava problemes
	% per afegir observadors (quan es generava el lazyvalue)
	% Ara només retornen la primera referència
	\subsection{Equals en Operation}%Quim
	% Quan es va fer en primer lloc el template Method per a operation es va crear un mètode
	% equals. Comparar funcions es indecidible com a problema, pel que es va haver d'eliminar
	\subsection{Sheet, put i get}%Sergi
	% Primer vam pensar que només seria necessari utilzar el "MaybeValue" com a retorn del get.
	% El problema ens va generar era que si ens portaven una casella inval·lida retornavem Empty,
	% pel que vam fer el refactor d'Either. En crear el façade i en necessitar les cells per a crear 
	% referències ens vam adonar d'haver fet malament l'anàlisi de requeriments dels mètodes,
	% pel que es va canviar a Cell i llança excepcions de NoSuchElementException quan no troba
	% la cel·la.
	% Es va pensar d'utilitzar Optional, peró llavors el put i el get tindrien diferents formes de comunicar
	% el mateix error, pel que es va recorrer a les excepcions
	\section{Conclusions}%Quim


\end{document}

