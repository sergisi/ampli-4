% !TeX spellcheck = ca
\documentclass{article}
\usepackage[utf8]{inputenc}
\usepackage{graphicx}
\usepackage{hyperref}
\usepackage{amsmath}
\usepackage{ amssymb }
\usepackage{tikz}
\usepackage{float}
\usepackage[simplified]{pgf-umlcd}
\usepackage{subfiles}
\usepackage{listings}
\usepackage[lighttt]{lmodern}
\usepackage{color}
\usepackage{array}
\usepackage{textcomp}
\graphicspath{ {img/} }

\usetikzlibrary{positioning,fit,calc,arrows.meta, shapes, snakes}
%Tot això hauria d'anar en un pkg, però no sé com és fa
\newcommand*{\assignatura}[1]{\gdef\1assignatura{#1}}
\newcommand*{\grup}[1]{\gdef\3grup{#1}}
\newcommand*{\professorat}[1]{\gdef\4professorat{#1}}
\renewcommand{\title}[1]{\gdef\5title{#1}}
\renewcommand{\author}[1]{\gdef\6author{#1}}
\renewcommand{\date}[1]{\gdef\7date{#1}}
\renewcommand{\baselinestretch}{1.5}
\renewcommand{\maketitle}{ %fa el maketitle de nou
	\begin{titlepage}
		\raggedright{UNIVERSITAT DE LLEIDA \\
			Escola Politècnica Superior \\
			Grau en Enginyeria Informàtica\\
			\1assignatura\\}
		\vspace{5cm}
		\centering\huge{\5title \\}
		\vspace{3cm}
		\large{\6author} \\
		\normalsize{\3grup}
		\vfill
		Professorat : \4professorat \\
		Data : \7date
\end{titlepage}}
%Emplenar a partir d'aquí per a fer el títol : no se com es fa el package
%S'han de renombrar totes, inclús date, si un camp es deixa en blanc no apareix

\title{SpreadSheet}
\author{Ian Palacín Aliana, Quim Picó Mora, Sergi Simón Balcells}
\date{17 de Maig}
\assignatura{Ampliació de Bases de dades i Enginyeria del Programari}
\professorat{JM. Gimeno Illa}
\grup{Prelab 2}

\renewcommand{\refname}{Referències}
%Comença el document
\begin{document}
	\maketitle
	\newpage
	\section{Introducció}%ian

	En aquest document consta les decisions d'implementació més importants i els errors comesos
	a l'hora de dissenyar la infraestructura d'un full de càlcul. Una de les prioritats del disseny
	és que aquest sigui el més mandrós possible. En la implementació se segueixen una sèrie de patrons
	de disseny que s'aniran mencionant.



	\section{Disseny}
	\subsection{Diagrama UML}%InteliJ
	\subsection{LiftA2}%Sergi
	% Operació de haskell
	% La idea es treure la necessitat dels ifs en la Operation (treure la lògica del MaybeValue)
	% S'ha implementat com un double dispatch
	 \subfile{lift/lifta2.tex}
	\subsection{Operation i el patró Strategy}
	% Template method quedava una mica estrany utilitzant liftA2
	% Strategy quedava més net
	\subfile{strat/strat.tex}
	\subsection{EitherValue}%Sergi
	% Utilitzar MaybeValue no encapsulava els errors que es poguessin generar de forma
	% dinàmica en Strategy.
	% Un malentès en el disseny de Sheet va fer fer un refactor per a EitherValue.
	% El refactor va ser senzil, ja que liftA2 es aplicable en Eithers (és un aplicatiu)
	\subfile{either/either.tex}
	\subsection{LazyValue a Cell}%Quim
	% Per a realitzar lazyValue en les cel·les se va fer servir el patró observable.
	% Cada vegada que es canvia d'expressió es canvien tots els observables per cridar
	% el metode només quan es necessari
Per tal de realitzar l'avaluació de les cel·les de forma mandrosa, s'ha decidit utilitzar el patró observable. S'ha definit sobre cada cel·la la variable lazyValue, la qual s'actualitza quan es canvia l'expressió d'una cel·la. A la vegada, es propaga una notificació al llarg dels observables que depenien d'aquesta cel·la, és a dir notifica als observables de les cel·les que contenien observadors sobre la cel·la que ha canviat perquè actualitzin el seu valor. De mateixa manera, aquestes cel·les notifiquen als observables de les cel·les que contenien observadors a aquestes cel·les que contenien observadors sobre la cel·la que ha canviat... Propagant així la notificació al llarg de totes les cel·les que depenien de la cel·la que ha canviat i fent que actualitzin el seu valor.
	\subsection{Sheet}%Ian
	Hi ha hagut dues decisions d'implementació
	rellevants en la classe Sheet, en primer lloc com
	convertir una adreça en format d'string (ex. "a1")
	a un format que puguem utilitzar
	, i en segon lloc la decisió sobre com i on guardar
	les cel·les.\\

	\begin{itemize}
	\item
	Per convertir l'adreça hem utilitzat les expressions
	regulars de Pattern, on primerament filtrem que la
	string estigui composta únicament d'una o més lletres
	seguides d'un o més nombres. Més tard validem que
	l'adreça obtinguda estigui dintre dels límits del full de càlcul.
	\item
	La creació de les cel·les és mandrosa, és a dir, només es creen quan són estrictament necessàries.
	S'ha decidit fer-ho així per evitar ocupar massa memòria en sheets grans, on realment la majoria
	de cel·les no s'arribarien a utilitzar. Tot i que accedir amb un array d'arrays sembla més natural,
	hem decidit utilitzar un HashMap perquè és molt transparent tenint en compte com l'usa l'usuari final.\\
	Alguns dels motius pels quals hem decidit utilitzar un HashMap en comptes d'un array d'arrays:
	\begin{itemize}
	\item Amb un array d'arrays les adreces s'haurien de
	parsejar molt més sovint, amb un HashMap podem guardar
	com  clau l'string directament
	\item Amb un array d'arrays l'estructura quedaria
	plena de nulls, mentre que amb un HashMap solament
	guardes el necessari.
	\item Sheet requeria els mètodes get i put, que ens
	semblaven més naturals d'un HashMap per l'estructura
	clau, valor.
	\end{itemize}
	\end{itemize}


	% La creació de les cel·les es lazy, es a dir, només es creen quan son estrictament necessaries,
	% per a evitar ocupar massa memória en sheets molt grans quan no s'utilitza la majoria de cel·les.
	% Tot i que accedir amb una Array d'arrays sembla més natural, un hashmap es molt transparent
	% per com l'utilitza l'usuari final
	\section{Problemes}
	\subsection{References}%Ian
	A \texttt{References} es guarda la cel·la referenciada i un set de cel·les a les quals referència.
	\texttt{References} primerament avaluava la cel·la referenciada i les referències d'aquesta,
	això va acabar donant problemes a l'hora d'afegir observadors (quan es generava el \texttt{lazyvalue}).
	Ara només retorna la primera referència, aconseguint així una propagació més controlada.

	% References calculaba referències de la seva referència, i això donava problemes
	% per afegir observadors (quan es generava el lazyvalue)
	% Ara només retornen la primera referència
	\subsection{Equals en Operation}%Quim
	% Quan es va fer en primer lloc el template Method per a operation es va crear un mètode
	% equals. Comparar funcions es indecidible com a problema, pel que es va haver d'eliminar
En primera instància es va decidir fer ús de \textit{Template Method} per a implementar la part corresponent 
a \texttt{Operation}, \texttt{Sum} i \texttt{Mult}. En aquesta instància vam generar el mètode equals per a
realitzar els tests Però, quan es va canviar el patró de \textit{Template Method} a \textit{Strategy},
ens vam adonar que treballar amb funcions anònimes ens donava problemes, ja que comparar 
funcions és un problema indecidible (el teorema de Cook és aplicable). Per tant, es va decidir prescindir 
d'aquest mètode. 
	\subsection{Sheet, put i get}%Sergi
	% Primer vam pensar que només seria necessari utilzar el "MaybeValue" com a retorn del get.
	% El problema ens va generar era que si ens portaven una casella inval·lida retornavem Empty,
	% pel que vam fer el refactor d'Either. En crear el façade i en necessitar les cells per a crear 
	% referències ens vam adonar d'haver fet malament l'anàlisi de requeriments dels mètodes,
	% pel que es va canviar a Cell i llança excepcions de NoSuchElementException quan no troba
	% la cel·la.
	% Es va pensar d'utilitzar Optional, peró llavors el put i el get tindrien diferents formes de comunicar
	% el mateix error, pel que es va recorrer a les excepcions
	\subfile{putget/putget.tex}
	\section{Conclusions}%Quim
Per la implementació d'aquesta solució s'ha fet ús de tres patrons de disseny diferents. El patró façana 
ha permès ficar una capa que simplifica i unifica la utilització del nostre disseny. Per altra banda 
el patró observador ens ha permès realitzar l'avaluació "reactiva" de les cel·les de forma senzilla. També 
l'\textit{Strategy Method}, que a falta de no poder treballar amb \textit{Template Method}, ens ha permès 
seguir treballant amb funcions anònimes a \texttt{Operation}. I finalment el patró \texttt{NullValue}, que l'hem 
adaptat una mica substituint-lo per un \texttt{EitherValue}.

\end{document}

