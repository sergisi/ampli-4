\begin{document}
	En la part de decisions del disseny, s'expressa que un malentès ens porta a canviar el 
	\texttt{MaybeValue} a un \texttt{EitherValue}. Aquest canvi en el disseny era causat per
	un malentès del mètode \texttt{get} de \texttt{Sheet}. El malentès va ser que retornaria aquest
	mètode. En primer lloc es va pensar de retornar el \texttt{MaybeValue}, ja que pensàvem que 
	no necessitaríem mai la cel·la en qüestió. Però, això ens portava a veure que una casella buida
	podia significar massa coses, pel que vam refactoritzar el codi portant-lo a utilitzar \textit{Either}.
	\\
	\\
	Més tard, ens vam adonar que era necessari retornar \texttt{Cell} per a poder crear \texttt{Reference},
	pel que aquest motiu de refactor va quedar obsolet. Llavors es va decidir si tornar \texttt{Optional}
	 en el get o fer el mateix que en el mètode \texttt{put}. En aquell moment es va decidir que seria més
	 lògic seguir les excepcions en els dos mètodes, donat que \textit{Sheet} era mutable i fer que el mètode
	 \texttt{put} retornés algun valor seria contraintuïtiu en l'usuari final.
\end{document}