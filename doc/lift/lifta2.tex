\begin{document}
	La idea de les pistes que donava la pràctica amb les operacions 
	\texttt{hasValue} i \texttt{get} en els \texttt{MaybeValues} 
	donaven la sensació que transportava la lògica de manipular \texttt{MaybeValues} als mètodes d'\texttt{Operation}.
	Donat que aquesta classe era una referència clara a Haskell, es va investigar quina seria la forma natural de fer-ho
	en aquest llenguatge. Les formes que es van trobar requerien fer \texttt{MaybeValue} un aplicatiu o una mònada. 
	El problema
	d'aquests era que el codi es tornava progressivament més difícil d'implementar, pel que es va decidir veure
	funcions més específiques. Pel nostre cas d'ús, només necessitem la signatura de 
	\texttt{(a-> b-> c) -> f a -> f b -> f c}, que és exactament la del mètode \texttt{liftA2} de Haskell.\\
	\\
	Per a poder implementar aquest mètode, es va utilitzar la tècnica d'un \textit{double dispatch}, 
	que ens permetia canviar
	el comportament del mètode sabent més informació per part del compilador sense necessitat d'incloure \textit{ifs}.
	D'aquesta forma, es té només dos possibilitats: si el paràmetre era \texttt{MaybeValue} o si aquest era 
	\texttt{SomeValue}, ja que en el cas que fos \texttt{NoValue} el 
	mètode es retornaria a si mateix sense cap comprovació
	addicional. En el cas en el qual els dos són \texttt{SomeValue} el
	 mètode retornaria la funció que ha estat passat com a 
	paràmetre. D'aquesta forma l'algorisme en \texttt{Operation} 
	feia el codi molt més simple i fàcil de llegir.
\end{document}